\chapter{\HelloWorldSectionName}
\label{sec:helloworld}

\RU{Продолжим, используя знаменитый пример из книги}
\EN{Let's use the famous example from the book}
\ES{Continuemos usando el famoso ejemplo del libro}
``The C programming Language''\cite{Kernighan:1988:CPL:576122}:

\lstinputlisting{patterns/01_helloworld/hw.c}

\section{x86}

\input{patterns/01_helloworld/MSVC_x86}
\ifdefined\IncludeGCC
\input{patterns/01_helloworld/GCC_x86}
\fi

\section{x86-64}
\input{patterns/01_helloworld/MSVC_x64}
\ifdefined\IncludeGCC
\input{patterns/01_helloworld/GCC_x64}
\fi

\ifdefined\IncludeGCC
\input{patterns/01_helloworld/GCC_one_more}
\fi
\ifdefined\IncludeARM
\section{ARM}
\label{sec:hw_ARM}

\index{\idevices}
\index{Raspberry Pi}
\index{Xcode}
\index{LLVM}
\index{Keil}
\RU{Для экспериментов с процессором ARM несколько компиляторов было использовано:}%
\EN{For my experiments with ARM processors, several compilers were used:}%
\ES{Para mis experimentos con procesadores ARM, se usaron varios compiladores:} 

\begin{itemize}
\item \RU{Популярный в embedded-среде}\EN{Popular in the embedded area}\ES{Popular en el entorno embebido} Keil Release 6/2013.

\item Apple Xcode 4.6.3 \EN{IDE}\ES{IDE} (\RU{с компилятором}\EN{with}\ES{con} LLVM-GCC 4.2\EN{ compiler}\ES{ como compilador}%
\footnote{\EN{It is indeed so: Apple Xcode 4.6.3 uses open-source GCC as front-end compiler and LLVM 
code generator}\RU{Это действительно так: Apple Xcode 4.6.3 использует опен-сорсный GCC как компилятор
переднего плана и кодогенератор LLVM}\ES{Es así: Apple Xcode 4.6.3 usa GCC de código abierto como compilador de frontend y LLVM como generador de código}}).

%\item GCC 4.8.1 (Linaro) (\RU{для}\EN{for}\ES{para} ARM64).
%
\item GCC 4.9 (Linaro) (\RU{для}\EN{for}\ES{para} ARM64), 
\RU{доступный в виде исполняемого файла для win32 на}\EN{available as win32-executables at}\ES{disponible como ejecutables para win32 en} 
\url{http://go.yurichev.com/17325}.

\end{itemize}

\RU{Везде в этой книге, если не указано иное, идет речь о 32-битном ARM (включая режимы Thumb и Thumb-2).}
\EN{32-bit ARM code is used (including Thumb and Thumb-2 modes) in all cases in this book, if not mentioned otherwise.}
\ES{En todo este libro, salvo que se indique lo contrario, se usa ARM de 32 bits (incluidos los modos Thumb y Thumb-2).}
\RU{Когда речь идет о 64-битном ARM, он называется здесь ARM64.}
\EN{When we talk about 64-bit ARM here, it is called ARM64 here.}
\ES{Cuando hablamos de ARM de 64 bits aquí, lo llamamos ARM64.}

% subsections
\subsection{\NonOptimizingKeilVI (\ARMMode)}

\RU{Для начала скомпилируем наш пример в Keil}\EN{Let's start by compiling our example in Keil}\ES{Para empezar, compilamos nuestro ejemplo en Keil}:

\begin{lstlisting}
armcc.exe --arm --c90 -O0 1.c 
\end{lstlisting}

\index{\IntelSyntax}
\RU{Компилятор \IT{armcc} генерирует листинг на ассемблере в формате Intel.}
\EN{The \IT{armcc} compiler produces assembly listings in Intel-syntax} 
\ES{El compilador \IT{armcc} produce listados en ensamblador con sintaxis Intel} 
\RU{Этот листинг содержит некоторые высокоуровневые макросы, связанные с ARM}%
\EN{but it has high-level ARM-processor related macros}\ES{, pero incluye macros de alto nivel relacionados con ARM}\footnote{
\RU{например, он показывает инструкции \PUSH/\POP, отсутствующие в режиме ARM}
\EN{e.g. ARM mode lacks \PUSH/\POP instructions}\ES{p. ej., muestra instrucciones \PUSH/\POP, ausentes en el modo ARM}}, 
\RU{а нам важнее увидеть инструкции \q{как есть}, так что посмотрим скомпилированный результат в \IDA.}
\EN{but it is more important for us to see the instructions \q{as is} so let's see the compiled result in \IDA.}
\ES{pero nos importa más ver las instrucciones tal cual, así que veamos el resultado compilado en \IDA.}

\begin{lstlisting}[caption=\NonOptimizingKeilVI (\ARMMode) \IDA]
.text:00000000             main
.text:00000000 10 40 2D E9    STMFD   SP!, {R4,LR}
.text:00000004 1E 0E 8F E2    ADR     R0, aHelloWorld ; "hello, world"
.text:00000008 15 19 00 EB    BL      __2printf
.text:0000000C 00 00 A0 E3    MOV     R0, #0
.text:00000010 10 80 BD E8    LDMFD   SP!, {R4,PC}

.text:000001EC 68 65 6C 6C+aHelloWorld  DCB "hello, world",0    ; DATA XREF: main+4
\end{lstlisting}

\RU{В вышеприведённом примере можно легко увидеть, что каждая инструкция имеет размер 4 байта.}
\EN{In the example, we can easily see each instruction has a size of 4 bytes.}
\ES{En el ejemplo anterior se ve fácilmente que cada instrucción mide 4 bytes.}
\RU{Действительно, ведь мы же компилировали наш код для режима ARM, а не Thumb.}
\EN{Indeed, we compiled our code for ARM mode, not for Thumb.}
\ES{Efectivamente, compilamos el código para modo ARM, no Thumb.}

\index{ARM!\Instructions!STMFD}
\index{ARM!\Instructions!POP}
\RU{Самая первая инструкция}\EN{The very first instruction}\ES{La primera instrucción}, \TT{STMFD SP!, \{R4,LR\}}\footnote{\ac{STMFD}}, 
\RU{работает как инструкция}\EN{works as an x86}\ES{funciona como una} \PUSH \RU{в x86}\EN{instruction}\ES{ de x86},
\RU{записывая значения двух регистров}\EN{writing the values of two registers}\ES{escribiendo los valores de dos registros}
(\Reg{4} \AndENRU \ac{LR}) \RU{в стек}\EN{into the stack}\ES{ en la pila}.
\RU{Действительно, в выдаваемом листинге на ассемблере компилятор \IT{armcc} для упрощения указывает здесь инструкцию}
\EN{Indeed, in the output listing from the \IT{armcc} compiler, for the sake of simplification, 
actually shows the} \TT{PUSH \{r4,lr\}}\EN{ instruction}\ES{ de hecho muestra la instrucción}.
\RU{Но это не совсем точно, инструкция \PUSH доступна только в режиме Thumb, поэтому,
во избежание путаницы, я предложил работать в \IDA}%
\EN{But that is not quite precise. The \PUSH instruction is only available in Thumb mode.
So, to make things less confusing, we're doing this in \IDA}.
\ES{Pero eso no es del todo preciso: \PUSH solo está disponible en modo Thumb; para evitar confusiones, lo veremos en \IDA}.

\RU{Итак, эта инструкция уменьшает \ac{SP}, чтобы он указывал на место в стеке, свободное для записи
новых значений, затем записывает значения регистров \Reg{4} и \ac{LR} 
по адресу в памяти, на который указывает измененный регистр \ac{SP}}%
\EN{This instruction \glspl{decrement} first the \ac{SP} so it points to the place in the stack
that is free for new entries, then it saves the values of the \Reg{4} and \ac{LR} registers at the address
stored in the modified \ac{SP}}\ES{Esta instrucción primero \glspl{decrement} el \ac{SP} para que apunte a un lugar libre de la pila y después guarda los valores de \Reg{4} y \ac{LR} en la dirección indicada por el \ac{SP} modificado}.

\RU{Эта инструкция, как и инструкция \PUSH в режиме Thumb, может сохранить в стеке одновременно несколько значений регистров, что может быть очень удобно}%
\EN{This instruction (like the \PUSH instruction in Thumb mode) is able to save several register values at once which can be very useful}. 
\ES{Esta instrucción (como \PUSH en modo Thumb) puede guardar varios registros a la vez, lo cual puede ser muy útil}. 
\RU{Кстати, такого в x86 нет}\EN{By the way, this has no equivalent in x86}.\ES{Por cierto, esto no existe en x86}.
\RU{Также следует заметить, что \TT{STMFD}~--- генерализация инструкции \PUSH (то есть расширяет её возможности), потому что может работать с любым регистром, а не только с \ac{SP}.}
\EN{It can also be noted that the \TT{STMFD} instruction is a generalization 
of the \PUSH instruction (extending its features), since it can work with any register, not just with \ac{SP}.}
\ES{También puede notarse que \TT{STMFD} es una generalización de \PUSH (amplía sus capacidades), porque puede trabajar con cualquier registro y no solo con \ac{SP}.}
\RU{Другими словами, \TT{STMFD} можно использовать для записи набора регистров в указанном месте памяти.}
\EN{In other words, \TT{STMFD} may be used for storing pack of registers at the specified memory address.}
\ES{En otras palabras, \TT{STMFD} puede usarse para almacenar un conjunto de registros en la dirección de memoria indicada.}

\index{\PICcode}
\index{ARM!\Instructions!ADR}
\RU{Инструкция}\EN{The}\ES{La} \TT{ADR R0, aHelloWorld}
\RU{прибавляет или отнимает значение регистра \ac{PC} к смещению, где хранится строка}
\EN{instruction adds or subtracts the value in the \ac{PC} register to the offset where the}
\TT{hello, world}\EN{ string is located}\ES{ instrucción suma o resta el valor de \ac{PC} al desplazamiento donde está la cadena}.
\RU{Причем здесь \ac{PC}, можно спросить}\EN{How is the \TT{PC} register used here, one might ask}?\ES{¿Cómo se usa aquí \ac{PC}?}
\RU{Притом, что это так называемый \q{\PICcode}}\EN{This is so-called \q{\PICcode}.}\ES{Se trata del llamado \q{\PICcode}.}
\footnote{
	\RU{Читайте больше об этом в соответствующем разделе}
	\EN{Read more about it in relevant section}\ES{Lee más sobre esto en la sección correspondiente}~(\myref{sec:PIC})
	}
\RU{он предназначен для исполнения будучи не привязанным к каким-либо адресам в памяти}%
\EN{Such code can be be executed at a non-fixed address in memory}\ES{Este código puede ejecutarse en una dirección no fija de memoria}.
\EN{In other words, this is \ac{PC}-relative addressing.}
\RU{Другими словами, это относительная от \ac{PC} адресация.}
\ES{En otras palabras, es direccionamiento relativo al \ac{PC}.}
\RU{В опкоде инструкции \TT{ADR} указывается разница между адресом этой инструкции и местом, где хранится строка}%
\EN{The \TT{ADR} instruction takes into account the difference between the address of this instruction and the address where the string is located}.\ES{El op\-code de \TT{ADR} codifica la diferencia entre la dirección de esta instrucción y el lugar donde está la cadena}.
\RU{Эта разница всегда будет постоянной, вне зависимости от того, куда был загружен \ac{OS} наш код}%
\EN{This difference (offset) is always to be the same, no matter at what address our code is loaded by the \ac{OS}}\ES{Esta diferencia (desplazamiento) siempre será la misma, independientemente de la dirección a la que el \ac{OS} cargue nuestro código}.
\RU{Поэтому всё, что нужно~--- это прибавить адрес текущей инструкции (из \ac{PC}), чтобы получить текущий абсолютный адрес нашей Си-строки}%
\EN{That's why all we need is to add the address of the current instruction (from \ac{PC}) in order to get the absolute memory address of our C-string}\ES{Por eso, lo único necesario es sumar la dirección de la instrucción actual (desde \ac{PC}) para obtener la dirección absoluta de nuestra cadena en C}.

\index{ARM!\Registers!Link Register}
\index{ARM!\Instructions!BL}
\RU{Инструкция} \TT{BL \_\_2printf}\footnote{Branch with Link}
\RU{вызывает функцию \printf}\EN{instruction calls the \printf function}. \ES{llama a la función \printf}. 
\RU{Работа этой инструкции состоит из двух фаз}%
\EN{Here's how this instruction works}\ES{Así es como funciona esta instrucción}: 
\begin{itemize}
\item
\RU{записать адрес после инструкции \TT{BL} (\TT{0xC}) в регистр \ac{LR}}%
\EN{store the address following the \TT{BL} instruction (\TT{0xC}) into the \ac{LR}}\ES{guardar la dirección siguiente a \TT{BL} (\TT{0xC}) en \ac{LR}};
\item
\RU{передать управление в \printf, записав адрес этой функции в регистр \ac{PC}}%
\EN{then pass the control to the \printf by writing its address into the \ac{PC} register}\ES{luego transferir el control a \printf escribiendo su dirección en el registro \ac{PC}}.
\end{itemize}

\RU{Ведь когда функция \printf закончит работу, нужно знать, куда вернуть управление, поэтому закончив работу, всякая функция передает управление по адресу, записанному в регистре \ac{LR}}%
\EN{When \printf finishes its execution it must have information about where it needs to return the control to.
That's why each function passes control to the address stored in the \ac{LR} register}\ES{Cuando \printf termina debe saber adónde volver; por eso toda función devuelve el control a la dirección guardada en \ac{LR}}.

\RU{В этом разница между \q{чистыми} \ac{RISC}-процессорами вроде ARM и \ac{CISC}-процессорами как x86,
где адрес возврата обычно записывается в стек}%
\EN{That is a difference between \q{pure} \ac{RISC}-processors like ARM and \ac{CISC}-processors like x86,
where the return address is usually stored on the stack}\footnote{\RU{Подробнее об этом будет описано в следующей главе}\EN{Read more about this in next section}\ES{Lee más sobre esto en la siguiente sección}~(\myref{sec:stack})}.

\RU{Кстати, 32-битный абсолютный адрес (либо смещение) невозможно закодировать в 32-битной инструкции \TT{BL}, в ней есть место только для 24-х бит}%
\EN{By the way, an absolute 32-bit address or offset cannot be encoded in the 32-bit \TT{BL} instruction because
it only has space for 24 bits}.\ES{Por cierto, una dirección absoluta de 32 bits (o un desplazamiento) no puede codificarse en la instrucción de 32 bits \TT{BL}, porque solo hay 24 bits disponibles}.
\RU{Поскольку все инструкции в режиме ARM имеют длину 4 байта (32 бита) и инструкции могут находится только по адресам кратным 4, то последние 2 бита (всегда нулевых) можно не кодировать.}
\EN{As we may remember, all ARM-mode instructions have a size of 4 bytes (32 bits).
Hence, they can only be located on 4-byte boundary addresses.
This implies that the last 2 bits of the instruction address (which are always zero bits) may be omitted.}
\ES{Como recordamos, todas las instrucciones en modo ARM miden 4 bytes (32 bits) y solo pueden situarse en direcciones múltiplos de 4; por ello, los 2 bits bajos de la dirección (siempre a 0) no se codifican.}
\RU{В итоге имеем 26 бит, при помощи которых можно закодировать}
\EN{In summary, we have 26 bits for offset encoding. This is enough to encode} $current\_PC \pm{} \approx{}32M$.\ES{En resumen, tenemos 26 bits para codificar el desplazamiento; esto basta para} $current\_PC \pm{} \approx{}32M$.

\index{ARM!\Instructions!MOV}
\RU{Следующая инструкция}\EN{Next, the}\ES{A continuación, la} \TT{MOV R0, \#0}\footnote{MOVe}
\RU{просто записывает 0 в регистр \Reg{0}}\EN{instruction just writes 0 into the \Reg{0} register}\ES{ simplemente escribe 0 en el registro \Reg{0}}.
\RU{Ведь наша Си-функция возвращает 0, а возвращаемое значение всякая функция оставляет в \Reg{0}}%
\EN{That's because our C-function returns 0 and the return value is to be placed in the \Reg{0} register}\ES{Nuestra función en C devuelve 0 y el valor de retorno debe ponerse en \Reg{0}}.

\index{ARM!\Registers!Link Register}
\index{ARM!\Instructions!LDMFD}
\index{ARM!\Instructions!POP}
\RU{Последняя инструкция}\EN{The last instruction}\ES{La última instrucción} \TT{LDMFD SP!, {R4,PC}}\footnote{\ac{LDMFD}}\RU{~--- это инструкция, обратная}\EN{ is an inverse instruction of}\ES{ es la inversa de} \TT{STMFD}. 
\RU{Она загружает из стека (или любого другого места в памяти) значения для сохранения их в \Reg{4} и \ac{PC}, увеличивая \glslink{stack pointer}{указатель стека} \ac{SP}.}
\EN{It loads values from the stack (or any other memory place) in order to save them into \Reg{4} and \ac{PC}, and \glslink{increment}{increments} the \gls{stack pointer} \ac{SP}.}
\ES{Carga desde la pila (o cualquier otra zona de memoria) los valores para guardarlos en \Reg{4} y \ac{PC}, e \glslink{increment}{incrementa} el puntero de pila \ac{SP}.}
\RU{Здесь она работает как аналог \POP}\EN{It works like \POP here}\ES{Aquí actúa como un \POP}.\\
N.B. \RU{Самая первая инструкция \TT{STMFD} сохранила в стеке \Reg{4} и \ac{LR}, а \IT{восстанавливаются} во время исполнения \TT{LDMFD} регистры \Reg{4} и \ac{PC}}%
\EN{The very first instruction \TT{STMFD} saved the \Reg{4} and \ac{LR} registers pair on the stack, but \Reg{4} and \ac{PC} are \IT{restored} during the \TT{LDMFD} execution}\ES{La primera instrucción \TT{STMFD} guardó en la pila el par de registros \Reg{4} y \ac{LR}, pero durante \TT{LDMFD} se \IT{restauran} \Reg{4} y \ac{PC}}.

\RU{Как мы уже знаем, в регистре \ac{LR} обычно сохраняется адрес места, куда нужно всякой функции вернуть управление}%
\EN{As we already know, the address of the place where each function must return control to is usually saved in the \ac{LR} register}\ES{Como ya sabemos, en \ac{LR} suele guardarse la dirección a la que debe devolver el control cada función}.
\RU{Самая первая инструкция сохраняет это значение в стеке, потому что наша функция \main позже будет сама пользоваться этим регистром в момент вызова \printf}%
\EN{The very first instruction saves its value in the stack because the same register will be used by our
\main function when calling \printf}\ES{La primera instrucción guarda ese valor en la pila, porque nuestra función \main usará ese registro al llamar a \printf}.
\RU{А затем, в конце функции, это значение можно сразу записать прямо в \ac{PC}, таким образом, передав управление туда, откуда была вызвана наша функция}%
\EN{In the function's end, this value can be written directly to the \ac{PC} register, thus passing control to where our function was called}\ES{Al final de la función, ese valor puede escribirse directamente en \ac{PC}, devolviendo el control al lugar desde el que fue llamada}.

\RU{Так как функция \main обычно самая главная в \CCpp, управление будет возвращено в загрузчик \ac{OS}, либо куда-то в \ac{CRT} 
или что-то в этом роде.}
\EN{Since \main is usually the primary function in \CCpp,
the control will be returned to the \ac{OS} loader or to a point in a \ac{CRT},
or something like that.}
\ES{Como \main suele ser la función principal en \CCpp, el control volverá al cargador del \ac{OS}, o a algún punto del \ac{CRT}, o similar.}

\EN{All that allows to omit \TT{BX LR} instruction at the end of the function.}
\RU{Всё это позволяет избавиться от инструкции \TT{BX LR} в самом конце функции.}
\ES{Todo esto permite omitir la instrucción \TT{BX LR} al final de la función.}

\index{ARM!DCB}
\TT{DCB}\RU{~--- директива ассемблера, описывающая массивы байт или ASCII-строк, аналог директивы DB в 
x86-ассемблере}%
\EN{~is an assembly language directive defining an array of bytes or ASCII strings, akin to the DB directive 
in x86-assembly language}\ES{~es una directiva del lenguaje ensamblador que define un array de bytes o cadenas ASCII, similar a la directiva DB en ensamblador x86}.


\subsection{\NonOptimizingKeilVI (\ThumbMode)}

\RU{Скомпилируем тот же пример в Keil для режима Thumb}\EN{Let's compile the same example using Keil in Thumb mode}\ES{Compilamos el mismo ejemplo en Keil en modo Thumb}:

\begin{lstlisting}
armcc.exe --thumb --c90 -O0 1.c 
\end{lstlisting}

\RU{Получим (в \IDA)}\EN{We are getting (in \IDA)}\ES{Obtenemos (en \IDA)}:

\begin{lstlisting}[caption=\NonOptimizingKeilVI (\ThumbMode) + \IDA]
.text:00000000             main
.text:00000000 10 B5          PUSH    {R4,LR}
.text:00000002 C0 A0          ADR     R0, aHelloWorld ; "hello, world"
.text:00000004 06 F0 2E F9    BL      __2printf
.text:00000008 00 20          MOVS    R0, #0
.text:0000000A 10 BD          POP     {R4,PC}

.text:00000304 68 65 6C 6C+aHelloWorld  DCB "hello, world",0    ; DATA XREF: main+2
\end{lstlisting}

\RU{Сразу бросаются в глаза двухбайтные (16-битные) опкоды\EMDASH{}это, как уже было отмечено, Thumb.}%
\EN{We can easily spot the 2-byte (16-bit) opcodes. This is, as was already noted, Thumb.}
\ES{Saltan a la vista los opcodes de 2 bytes (16 bits); como ya se señaló, es Thumb.}
\index{ARM!\Instructions!BL}
\RU{Кроме инструкции \TT{BL}.}\EN{The \TT{BL} instruction, however, }\ES{Excepto la instrucción \TT{BL}.}
\RU{Но на самом деле она состоит из двух 16-битных инструкций}%
\EN{consists of two 16-bit instructions}.\ES{En realidad consta de dos instrucciones de 16 bits}.
\RU{Это потому что в одном 16-битном опкоде слишком мало места для задания смещения, по которому находится функция \printf}%
\EN{This is because it is impossible to load an offset for the \printf function while using the small space in one 16-bit opcode}.\ES{Esto es porque en un solo op\-code de 16 bits hay muy poco espacio para especificar el desplazamiento hasta la función \printf}.
\RU{Так что первая 16-битная инструкция загружает старшие 10 бит смещения, а вторая~--- младшие 11 бит смещения}%
\EN{Therefore, the first 16-bit instruction loads the higher 10 bits of the offset and the second instruction loads 
the lower 11 bits of the offset}.\ES{Así que la primera instrucción de 16 bits carga los 10 bits altos del desplazamiento y la segunda los 11 bits bajos}.
\RU{Как уже было упомянуто, все инструкции в Thumb-режиме имеют длину 2 байта (или 16 бит)}%
\EN{As was noted, all instructions in Thumb mode have a size of 2 bytes (or 16 bits)}.
\ES{Como se mencionó, todas las instrucciones en modo Thumb tienen longitud de 2 bytes (o 16 bits).}
\RU{Поэтому невозможна такая ситуация, когда Thumb-инструкция начинается по нечетному адресу.}
\EN{This implies it is impossible for a Thumb-instruction to be at an odd address whatsoever.}
\ES{Por tanto, es imposible que una instrucción Thumb empiece en una dirección impar.}
\RU{Учитывая сказанное, последний бит адреса можно не кодировать}%
\EN{Given the above, the last address bit may be omitted while encoding instructions}.
\ES{Dado lo anterior, el último bit de la dirección puede omitirse al codificar instrucciones}.
\RU{Таким образом, в Thumb-инструкции \TT{BL} можно закодировать адрес}
\EN{In summary, \TT{BL} Thumb-instruction can encode the address} $current\_PC \pm{}\approx{}2M$.\ES{En resumen, en la instrucción Thumb \TT{BL} se puede codificar la dirección} $current\_PC \pm{}\approx{}2M$.

\index{ARM!\Instructions!PUSH}
\index{ARM!\Instructions!POP}
\RU{Остальные инструкции в функции (\PUSH и \POP) здесь работают почти так же, как и описанные \TT{STMFD}/\TT{LDMFD}, только регистр \ac{SP} здесь не указывается явно}%
\EN{As for the other instructions in the function: \PUSH and \POP work here just like the described \TT{STMFD}/\TT{LDMFD} only the \ac{SP} register is not mentioned explicitly here}.
\ES{El resto de instrucciones de la función (\PUSH y \POP) funcionan aquí casi igual que las \TT{STMFD}/\TT{LDMFD} descritas; solo que el registro \ac{SP} no se indica explícitamente}.
\TT{ADR} \RU{работает так же, как и в предыдущем примере}\EN{works just like in the previous example}\ES{funciona igual que en el ejemplo anterior}.
\TT{MOVS} \RU{записывает 0 в регистр \Reg{0} для возврата нуля}%
\EN{writes 0 into the \Reg{0} register in order to return zero}.\ES{escribe 0 en el registro \Reg{0} para devolver cero}.

\subsection{\OptimizingXcodeIV (\ARMMode)}

Xcode 4.6.3 \RU{без включенной оптимизации выдает слишком много лишнего кода, поэтому включим оптимизацию компилятора (ключ \Othree), потому что там меньше инструкций.}
\EN{without optimization turned on produces a lot of redundant code so we'll study optimized output, where the instruction count is as small as possible, setting the compiler switch \Othree.}
\ES{sin optimización genera demasiado código redundante, así que estudiaremos la salida optimizada, donde hay el menor número de instrucciones posible, activando el conmutador del compilador \Othree.}

\begin{lstlisting}[caption=\OptimizingXcodeIV (\ARMMode)]
__text:000028C4             _hello_world
__text:000028C4 80 40 2D E9   STMFD           SP!, {R7,LR}
__text:000028C8 86 06 01 E3   MOV             R0, #0x1686
__text:000028CC 0D 70 A0 E1   MOV             R7, SP
__text:000028D0 00 00 40 E3   MOVT            R0, #0
__text:000028D4 00 00 8F E0   ADD             R0, PC, R0
__text:000028D8 C3 05 00 EB   BL              _puts
__text:000028DC 00 00 A0 E3   MOV             R0, #0
__text:000028E0 80 80 BD E8   LDMFD           SP!, {R7,PC}

__cstring:00003F62 48 65 6C 6C+aHelloWorld_0  DCB "Hello world!",0
\end{lstlisting}

\RU{Инструкции}\EN{The instructions}\ES{Las instrucciones} \TT{STMFD} \AndENRU \TT{LDMFD} \RU{нам уже знакомы}\EN{are already familiar to us}\ES{ya nos son familiares}.

\index{ARM!\Instructions!MOV}
\RU{Инструкция \MOV просто записывает число \TT{0x1686} в регистр \Reg{0}~--- это смещение, указывающее на строку \q{Hello world!}}%
\EN{The \MOV instruction just writes the number \TT{0x1686} into the \Reg{0} register.
This is the offset pointing to the \q{Hello world!} string}.\ES{La instrucción \MOV simplemente escribe el número \TT{0x1686} en \Reg{0}: es el desplazamiento que apunta a la cadena \q{Hello world!}}.

\RU{Регистр \TT{R7} (по стандарту, принятому в \cite{IOSABI}) это frame pointer, о нем будет рассказано позже.}
\EN{The \TT{R7} register (as it is standardized in \cite{IOSABI}) is a frame pointer. More on that below.}
\ES{El registro \TT{R7} (según el estándar de \cite{IOSABI}) es el frame pointer; lo veremos más abajo.}

\index{ARM!\Instructions!MOVT}
\RU{Инструкция}\EN{The} \TT{MOVT R0, \#0} (MOVe Top) \RU{записывает 0 в старшие 16 бит регистра}%
\EN{instruction writes 0 into higher 16 bits of the register}.\ES{ escribe 0 en los 16 bits altos del registro}.
\RU{Дело в том, что обычная инструкция \MOV в режиме ARM может записывать какое-либо значение только в младшие 16 бит регистра, ведь в ней нельзя закодировать больше}%
\EN{The issue here is that the generic \MOV instruction in ARM mode may write only the lower 16 bits of the register}.\ES{ El asunto es que la \MOV genérica en modo ARM solo puede escribir en los 16 bits bajos del registro}.
\RU{Помните, что в режиме ARM опкоды всех инструкций ограничены длиной в 32 бита. Конечно, это ограничение не касается перемещений данных между регистрами.}
\EN{Remember, all instruction opcodes in ARM mode are limited in size to 32 bits. Of course, this limitation is not related to moving data between registers.}
\ES{Recuerda que en modo ARM todos los opcodes están limitados a 32 bits. Claro que esto no afecta a los movimientos de datos entre registros.}
\RU{Поэтому для записи в старшие биты (с 16-го по 31-й включительно) существует дополнительная команда \TT{MOVT}}%
\EN{That's why an additional instruction \TT{MOVT} exists for writing into the higher bits (from 16 to 31 inclusive)}.\ES{ Por eso existe la instrucción adicional \TT{MOVT} para escribir en los bits altos (del 16 al 31 inclusive)}.
\RU{Впрочем, здесь её использование избыточно, потому что инструкция \TT{MOV R0, \#0x1686} выше и так обнулила старшую часть регистра}%
\EN{Its usage here, however, is redundant because the \TT{MOV R0, \#0x1686} instruction above cleared the higher part of the register}.\ES{ Su uso aquí es redundante, porque \TT{MOV R0, \#0x1686} ya limpió la parte alta del registro}.
\RU{Возможно, это недочет компилятора}\EN{This is probably a shortcoming of the compiler}.\ES{ Probablemente sea una limitación del compilador}.

\index{ARM!\Instructions!ADD}
\RU{Инструкция}\EN{The} \TT{ADD R0, PC, R0} \RU{прибавляет \ac{PC} к \Reg{0} для вычисления действительного адреса строки \q{Hello world!}. Как нам уже известно, это \q{\PICcode}, поэтому такая корректива необходима}%
\EN{instruction adds the value in the \ac{PC} to the value in the \Reg{0}, to calculate absolute address of the \q{Hello world!} string. 
As we already know, it is \q{\PICcode} so this correction is essential here}.\ES{ suma \ac{PC} a \Reg{0} para calcular la dirección absoluta de la cadena \q{Hello world!}. Como ya sabemos, es \q{\PICcode}, así que esta corrección es esencial}.

\RU{Инструкция \TT{BL} вызывает \puts вместо \printf}%
\EN{The \TT{BL} instruction calls the \puts function instead of \printf}.\ES{ La instrucción \TT{BL} llama a \puts en lugar de \printf}.

\label{puts}
\index{\CStandardLibrary!puts()}
\index{puts() \RU{вместо}\EN{instead of}\ES{en lugar de} printf()}
\RU{Компилятор заменил вызов \printf на \puts. 
Действительно, \printf с одним аргументом это почти аналог \puts.}
\EN{GCC replaced the first \printf call with \puts.
Indeed: \printf with a sole argument is almost analogous to \puts.} 
\ES{El compilador sustituyó \printf por \puts. En efecto, \printf con un único argumento es casi equivalente a \puts.} 
\RU{\IT{Почти}, если принять условие, что в строке не будет управляющих символов \printf, 
начинающихся со знака процента. Тогда эффект от работы этих двух функций будет разным}%
\EN{\IT{Almost}, because the two functions are producing the same result only in case the 
string does not contain printf format identifiers starting with \IT{\%}. 
In case it does, the effect of these two functions would be different}%
\ES{\IT{Casi}, porque ambas funciones producen el mismo resultado solo si la cadena no contiene especificadores de formato de \printf que empiezan por \IT{\%}. Si los contiene, el efecto será diferente}%
\footnote{
\RU{Также нужно заметить, что \puts не требует символа перевода строки `\textbackslash{}n' в конце строки,
поэтому его здесь нет.}
\EN{It has also to be noted the \puts does not require a `\textbackslash{}n' new line symbol 
at the end of a string, so we do not see it here.}\ES{También hay que notar que \puts no requiere el símbolo de nueva línea `\textbackslash{}n' al final de la cadena, por eso aquí no aparece.}}.

\RU{Зачем компилятор заменил один вызов на другой? Наверное потому что \puts работает быстрее}%
\EN{Why did the compiler replace the \printf with \puts? Probably because \puts is faster}%
\ES{¿Por qué el compilador sustituyó \printf por \puts? Probablemente porque \puts es más rápida}%
\footnote{\href{http://go.yurichev.com/17063}{ciselant.de/projects/gcc\_printf/gcc\_printf.html}}. 
\RU{Видимо потому что \puts проталкивает символы в \gls{stdout} не сравнивая каждый со знаком процента.}
\EN{Because it just passes characters to \gls{stdout} without comparing every one of them with the \IT{\%} symbol.}
\ES{Seguramente porque \puts envía los caracteres a \gls{stdout} sin comparar cada uno con el símbolo \IT{\%}.}

\RU{Далее уже знакомая инструкция}\EN{Next, we see the familiar}\ES{A continuación, vemos la conocida} 
\TT{MOV R0, \#0}\RU{, служащая для установки в 0 возвращаемого значения функции}%
\EN{instruction intended to set the \Reg{0} register to 0}\ES{ instrucción para establecer a 0 el valor devuelto en \Reg{0}}.

\input{patterns/01_helloworld/ARM/xcode_T2}
\subsection{ARM64}

\subsubsection{GCC}

\RU{Компилируем пример в}\EN{Let's compile the example using}\ES{Compilamos el ejemplo con} GCC 4.8.1 \InENRU ARM64:

\lstinputlisting[numbers=left,label=hw_ARM64_GCC,caption=\NonOptimizing GCC 4.8.1 + objdump]
{patterns/01_helloworld/ARM/hw.lst}

\RU{В ARM64 нет режима Thumb и Thumb-2, только ARM, так что тут только 32-битные инструкции.}
\EN{There are no Thumb and Thumb-2 modes in ARM64, only ARM, so there are 32-bit instructions only.}
\ES{En ARM64 no existen los modos Thumb ni Thumb-2, solo ARM, así que solo hay instrucciones de 32 bits.}
\RU{Регистров тут в 2 раза больше}\EN{Registers count is doubled}\ES{La cantidad de registros se duplica}: \myref{ARM64_GPRs}.
\RU{64-битные регистры теперь имеют префикс}\EN{64-bit registers has}\ES{Los registros de 64 bits tienen el prefijo} 
\TT{X-}\EN{ prefixes, while its 32-bit parts}\RU{, а их 32-битные части}\EMDASH{}\TT{W-}\ES{, mientras que sus partes de 32 bits — W-}.

\index{ARM!\Instructions!STP}
\EN{The }\RU{Инструкция }\TT{STP}\EN{ instruction} (\IT{Store Pair}) 
\RU{сохраняет в стеке сразу два регистра}\EN{saves two registers in the stack simultaneously}\ES{guarda dos registros en la pila simultáneamente}: \RegX{29} \InENRU \RegX{30}.
\RU{Конечно, эта инструкция может сохранять эту пару где угодно в памяти, но здесь указан регистр \ac{SP}, так что
пара сохраняется именно в стеке.}
\EN{Of course, this instruction is able to save this pair at a random place of memory, 
but the \ac{SP} register is specified here, so the pair is saved in the stack.}
\ES{Por supuesto, esta instrucción puede guardar ese par en cualquier lugar de la memoria, pero aquí se especifica el registro \ac{SP}, así que el par se guarda en la pila.}
\RU{Регистры в ARM64 64-битные, каждый имеет длину в 8 байт, так что для хранения двух регистров нужно именно 16 байт.}
\EN{ARM64 registers are 64-bit ones, each has a size of 8 bytes, so one needs 16 bytes for saving two registers.}
\ES{Los registros en ARM64 son de 64 bits y cada uno ocupa 8 bytes, por lo que se necesitan 16 bytes para guardar dos registros.}

\RU{Восклицательный знак после операнда означает, что сначала от \ac{SP} будет отнято 16 и только затем
значения из пары регистров будут записаны в стек.}
\EN{Exclamation mark after operand mean that 16 is to be subtracted from \ac{SP} first, and only then
values from registers pair are to be written into the stack.}
\ES{El signo de exclamación tras el operando significa que primero se resta 16 de \ac{SP} y solo después los valores del par de registros se escriben en la pila.}
\RU{Это называется}\EN{This is also called}\ES{A esto también se le llama} \IT{pre-index}.
\RU{Больше о разнице между}\EN{About the difference between}\ES{Sobre la diferencia entre} \IT{post-index} \AndENRU \IT{pre-index} 
\RU{описано здесь}\EN{read here}\ES{lee aquí}: \myref{ARM_postindex_vs_preindex}.

\RU{Таким образом, в терминах более знакомого всем процессора x86, первая инструкция~--- это просто аналог 
пары инструкций}
\EN{Hence, in the terms of more familiar x86, the first instruction is just an analogue to pair of}
\ES{Por lo tanto, en términos del más familiar x86, la primera instrucción es análoga a un par de}
\TT{PUSH X29} \AndENRU \TT{PUSH X30}.
\RegX{29} \EN{is used as \ac{FP} in ARM64}\RU{в ARM64 используется как \ac{FP}}, \EN{and}\RU{а} \RegX{30} 
\EN{as}\RU{как} \ac{LR}, \RU{поэтому они сохраняются в прологе функции и
восстанавливаются в эпилоге}\EN{so that's why they are saved in the function prologue and restored in the function epilogue}\ES{; por eso se guardan en el prólogo de la función y se restauran en el epílogo}.

\EN{The second instruction copies}\RU{Вторая инструкция копирует}\ES{La segunda instrucción copia} \ac{SP} \InENRU \RegX{29} (\OrENRU \ac{FP}).
\RU{Это нужно для установки стекового фрейма функции}\EN{This is done to set up the function stack frame}\ES{Esto se hace para establecer el marco de pila de la función}.

\label{pointers_ADRP_and_ADD}
\index{ARM!\Instructions!ADRP/ADD pair}
\RU{Инструкции }\TT{ADRP} \AndENRU \ADD \EN{instructions are used to fill the 
string}\RU{нужны для формирования адреса строки}\ES{se usan para formar la dirección de la cadena} \q{Hello!} \EN{address into the \RegX{0} register}\RU{в регистре \RegX{0}}\ES{ en el registro \RegX{0}}, 
\RU{ведь первый аргумент функции передается через этот регистр}\EN{because the first function argument is passed
in this register}\ES{porque el primer argumento de la función se pasa por ese registro}.
\RU{Но в ARM нет инструкций, при помощи которых можно записать в регистр длинное число}\EN{There are
no instructions, whatsoever, in ARM that can store a large number into a register} 
(\RU{потому что сама длина инструкции ограничена 4-я байтами. Больше об этом здесь}\EN{because the instruction
length is limited to 4 bytes, read more about it here}\ES{porque la longitud de la instrucción está limitada a 4 bytes; más sobre esto aquí}: \myref{ARM_big_constants_loading}).
\RU{Так что нужно использовать несколько инструкций}\EN{So several instructions must be utilised}\ES{Así que deben usarse varias instrucciones}.
\RU{Первая инструкция}\EN{The first instruction}\ES{La primera instrucción} (\TT{ADRP}) \EN{writes address of 4Kb page where the string is
located into \RegX{0}}\RU{записывает в \RegX{0} адрес 4-килобайтной страницы где находится строка}\ES{escribe en \RegX{0} la dirección de la página de 4 KB donde está la cadena}, 
\EN{and the the second one}\RU{а вторая}\ES{y la segunda} (\ADD) \RU{просто прибавляет к этому адресу остаток}\EN{just adds
reminder to the address}\ES{simplemente suma el resto a esa dirección}.
\EN{More about that in}\RU{Читайте больше об этом}\ES{Más sobre esto en}: \myref{ARM64_relocs}.

\TT{0x400000 + 0x648 = 0x400648}, \EN{and we see our \q{Hello!} C-string in the \TT{.rodata} data segment at this
address}\RU{и мы видим, что в секции данных \TT{.rodata} по этому адресу как раз находится наша
Си-строка \q{Hello!}}\ES{y vemos que en la sección de datos \TT{.rodata} en esa dirección está nuestra cadena en C «Hello!»}.

\index{ARM!\Instructions!BL}
\RU{Затем при помощи инструкции \TT{BL} вызывается \puts. Это уже рассматривалось ранее: \myref{puts}.}
\EN{\puts is called afterwards using \TT{BL} instruction. This was already discussed: \myref{puts}.}
\ES{Luego se llama a \puts usando la instrucción \TT{BL}. Esto ya se comentó: \myref{puts}.}

\RU{Инструкция }\MOV \EN{instruction writes 0 into}\RU{записывает 0 в}\ES{La instrucción }\MOV \ES{escribe 0 en} \RegW{0}. 
\RegW{0} \RU{это младшие 32 бита 64-битного регистра}\EN{is low 32 bits of 64-bit}\ES{son los 32 bits bajos del registro de 64 bits} \RegX{0}\EN{ register}:

\input{ARM_X0_register}

\RU{А результат функции возвращается через \RegX{0}, и \main возвращает 0, 
так что вот так готовится возвращаемый результат.}
\EN{The function result is returned via \RegX{0} and \main returns 0, so that's how the return
result is prepared.}
\ES{El resultado de la función se devuelve por \RegX{0}, y \main devuelve 0, así que así se prepara el valor de retorno.}
\RU{Почему именно 32-битная часть}\EN{But why using the 32-bit part}\ES{¿Por qué usar la parte de 32 bits}?
\RU{Потому в ARM64, как и в x86-64, тип \Tint оставили 32-битным, для лучшей совместимости.}
\EN{Because \Tint data type in ARM64, just like in x86-64, is still 32-bit, for better compatibility.}
\ES{Porque el tipo \Tint en ARM64, igual que en x86-64, sigue siendo de 32 bits para una mejor compatibilidad.}
\RU{Следовательно, раз уж функция возвращает 32-битный \Tint, то нужно заполнить только 32 младших бита 
регистра \RegX{0}.}
\EN{So if a function returns 32-bit \Tint, only the low 32 bits of \RegX{0} register has to be filled.}
\ES{Por lo tanto, si una función devuelve un \Tint de 32 bits, solo hay que llenar los 32 bits bajos del registro \RegX{0}.}

\RU{Для того, чтобы удостовериться в этом, немного отредактируем этот пример и перекомпилируем его.}%
\EN{In order to verify this, let's change this example slightly and recompile it.}
\ES{Para verificarlo, modifiquemos un poco este ejemplo y recompilémoslo.}
\RU{Теперь \main возвращает 64-битное значение:}%
\EN{Now \main returns 64-bit value:}
\ES{Ahora \main devuelve un valor de 64 bits:}

\begin{lstlisting}[caption=\main \RU{возвращающая значение типа}\EN{returning a value of}\ES{que devuelve un valor de tipo} \TT{uint64\_t}\EN{ type}\ES{ }]
#include <stdio.h>
#include <stdint.h>

uint64_t main()
{
        printf ("Hello!\n");
        return 0;
}
\end{lstlisting}

\RU{Результат точно такой же, только \MOV в той строке теперь выглядит так:}
\EN{The result is the same, but that's how \MOV at that line looks like now:}
\ES{El resultado es el mismo, solo que así se ve ahora el \MOV en esa línea:}

\begin{lstlisting}[caption=\NonOptimizing GCC 4.8.1 + objdump]
  4005a4:       d2800000        mov     x0, #0x0                        // #0
\end{lstlisting}

\index{ARM!\Instructions!LDP}
\RU{Далее при помощи инструкции \TT{LDP} (\IT{Load Pair}) восстанавливаются регистры \RegX{29} и \RegX{30}.}
\EN{\TT{LDP} (\IT{Load Pair}) then restores \RegX{29} and \RegX{30} registers.}
\ES{Luego, con la instrucción \TT{LDP} (\IT{Load Pair}) se restauran los registros \RegX{29} y \RegX{30}.}
\RU{Восклицательного знака после инструкции нет. Это означает, что сначала значения достаются из стека,
и только потом \ac{SP} увеличивается на 16.}
\EN{There is no exclamation mark after the instruction: this implies that the value is first loaded from the stack,
only then \ac{SP} it is increased by 16.}
\ES{No hay signo de exclamación tras la instrucción: esto implica que primero se cargan los valores de la pila y solo después \ac{SP} se incrementa en 16.}
\RU{Это называется}\EN{This is called}\ES{Esto se llama} \IT{post-index}.

\index{ARM!\Instructions!RET}
\RU{В ARM64 есть новая инструкция}\EN{New instruction appeared in ARM64}\ES{En ARM64 apareció una instrucción nueva}: \RET. 
\RU{Она работает так же как и}\EN{It works just as}\ES{Funciona igual que} \TT{BX LR}, \RU{но там добавлен специальный бит,
подсказывающий процессору, что это именно выход из функции, а не просто переход, чтобы процессор
мог более оптимально исполнять эту инструкцию}\EN{only a special \IT{hint} bit is added, informing the \ac{CPU}
that this is a return from a function, not just another jump instruction, so it can execute it more optimally}\ES{, solo que se añade un bit de \IT{pista} especial que informa a la \ac{CPU} de que es un retorno de función, no un salto cualquiera, para poder ejecutarla de forma más óptima}.

\RU{Из-за простоты этой функции оптимизирующий GCC генерирует точно такой же код.}
\EN{Due to the simplicity of the function, optimizing GCC generates the very same code.}
\ES{Debido a la simplicidad de la función, el GCC con optimización genera exactamente el mismo código.}


\fi
\ifdefined\IncludeMIPS
\chapter{
\RU{Простейшая функция}
\EN{The simplest Function}
\ES{La función más simple}
\PTBR{Brazilian portuguese text here}
}

\RU{Наверное, простейшая из возможных функций это та что возвращает некоторую константу:}%
\EN{The simplest possible function is arguably one that simply returns a constant value:}
\ES{La función más simple posible es, probablemente, aquella que simplemente devuelve un valor constante:}

\RU{Вот, например}\EN{Here it is}:
\ES{Aquí está}:

\lstinputlisting[caption=\EN{\CCpp Code}\RU{Код на \CCpp}\ES{Código \CCpp}]{patterns/00_ret/1.c}

\RU{Скомпилируем её!}
\EN{Lets compile it!}
\ES{¡Compilémosla!}

\section{x86}

\RU{И вот что делает оптимизирующий GCC}\EN{Here's what both the optimizing GCC and MSVC compilers produce on the x86 platform}:
\ES{Esto es lo que producen los compiladores GCC y MSVC con optimización en la plataforma x86}:

\lstinputlisting[caption=\Optimizing GCC/MSVC (\assemblyOutput)]{patterns/00_ret/1.s}

\index{x86!\Instructions!RET}
\RU{Здесь только две инструкции. Первая помещает значение 123 в регистр \EAX, который используется
для передачи возвращаемых значений. Вторая это \RET, которая возвращает управление в вызывающую функцию.}
\EN{There are just two instructions: the first places the value 123 into the \EAX register, which is used by convention for storing the return
value and the second one is \RET, which returns execution to the \gls{caller}.}
\ES{Solo hay dos instrucciones: la primera coloca el valor 123 en el registro \EAX, que por convención se usa para almacenar el valor de retorno, y la segunda es \RET, que devuelve la ejecución a la \gls{caller}.}
\RU{Вызывающая функция возьмет результат из регистра \EAX.}
\EN{The caller will take the result from the \EAX register.}
\ES{La función llamadora tomará el resultado del registro \EAX.}

\ifdefined\IncludeARM
\section{ARM}

\RU{А что насчет ARM?}\EN{There are a few differences on the ARM platform:}
\ES{Hay algunas diferencias en la plataforma ARM:}

\lstinputlisting[caption=\OptimizingKeilVI (\ARMMode) ASM Output]{patterns/00_ret/1_Keil_ARM_O3.s}

\RU{ARM использует регистр \Reg{0} для возврата значений, так что здесь 123 помещается в \Reg{0}.}
\EN{ARM uses the register \Reg{0} for returning the results of functions, so 123 is copied into \Reg{0}.}
\ES{ARM utiliza el registro \Reg{0} para devolver los resultados de las funciones, por lo que 123 se copia en \Reg{0}.}

\RU{Адрес возврата (\ac{RA}) в ARM не сохраняется в локальном стеке, а в регистре \ac{LR}.
Так что инструкция \TT{BX LR} делает переход по этому адресу, и это то же самое что и вернуть управление
в вызывающую ф-цию.}
%Maybe explain what a link register is, or if it is just a normal register, say so?
\EN{The return address is not saved on the local stack in the ARM \ac{ISA}, but rather in the link register, 
so the \TT{BX LR} instruction causes execution to jump to that address\EMDASH{}effectively returning execution to the \gls{caller}.}
\ES{En ARM la dirección de retorno no se guarda en la pila local, sino en el registro de enlace (\ac{LR}). Por ello, la instrucción \TT{BX LR} salta a esa dirección, lo que equivale a devolver la ejecución a la \gls{caller}.}
\fi

\index{ARM!\Instructions!MOV}
\index{x86!\Instructions!MOV}
\RU{Нужно отметить, что название инструкции \MOV в x86 и ARM сбивает с толку.}
\EN{It worth noting that \MOV is a misleading name for the instruction in both x86 and ARM \ac{ISA}s. }
\RU{На самом деле, данные не \IT{перемещаются}, а скорее \IT{копируются}.}
\EN{The data is not in fact \IT{moved}, but \IT{copied}.}
\ES{Vale la pena señalar que \MOV es un nombre equívoco para la instrucción tanto en x86 como en las \ac{ISA} ARM.}
\ES{En realidad, los datos no se \IT{mueven}, sino que se \IT{copian}.}

\ifdefined\IncludeMIPS
\section{MIPS}

\label{MIPS_leaf_function_ex1}
\RU{Есть два способа называть регистры в мире MIPS.}
\EN{There are two naming conventions used in the world of MIPS when naming registers:}
\ES{Hay dos convenciones para nombrar registros en MIPS.}
\RU{По номеру (от \$0 до \$31) или по псевдоимени (\$V0, \$A0, \etc{}.).}
\EN{by number (from \$0 to \$31) or by pseudoname (\$V0, \$A0, \etc{}).}
\ES{por número (de \$0 a \$31) o por seudónimo (\$V0, \$A0, \etc{}).}
\RU{Вывод на ассемблере в GCC показывает регистры по номерам:}
\EN{The GCC assembly output below lists registers by number:}
\ES{La salida de ensamblador de GCC a continuación muestra los registros por número:}

\lstinputlisting[caption=\Optimizing GCC 4.4.5 (\assemblyOutput)]{patterns/00_ret/MIPS.s}

\dots \RU{а \IDA\EMDASH{}по псевдоименам}\EN{while \IDA does it\EMDASH{}by their pseudonames}:
\ES{… mientras que \IDA los muestra\EMDASH{}por sus seudónimos}:

\lstinputlisting[caption=\Optimizing GCC 4.4.5 (IDA)]{patterns/00_ret/MIPS_IDA.lst}

\RU{Так что регистр \$2 (или \$V0) используется для возврата значений.}
\EN{The \$2 (or \$V0) register is used to store the function's return value.}
\ES{Así que el registro \$2 (o \$V0) se usa para almacenar el valor de retorno de la función.}
\index{MIPS!\Pseudoinstructions!LI}
LI \RU{это}\EN{stands for} ``Load Immediate'' \EN{and is the MIPS equivalent to MOV}.
\ES{significa “Load Immediate” y es el equivalente a MOV en MIPS}.

\index{MIPS!\Instructions!J}
\RU{Другая инструкция это инструкция перехода (J или JR), которая возвращает управление в 
\glslink{caller}{вызывающую ф-цию}, переходя по адресу в регистре \$31 (или \$RA).}
\EN{The other instruction is the jump instruction (J or JR) which returns the execution flow to the \gls{caller},
jumping to the address in the \$31 (or \$RA) register.}
\RU{Это аналог регистра \ac{LR} в ARM.}
\EN{This is the register analogous to \ac{LR} in ARM.}
\ES{La otra instrucción es la de salto (J o JR), que devuelve la ejecución a la \gls{caller} saltando a la dirección contenida en el registro \$31 (o \$RA).}
\ES{Este registro es análogo al \ac{LR} en ARM.}

\RU{Но почему инструкция загрузки (LI) и инструкция перехода (J или JR) поменены местами?}
\index{MIPS!Branch delay slot}
\RU{Это артефакт \ac{RISC} и называется он}
\EN{You might be wondering why positions of the the load instruction (LI) and the jump instruction (J or JR) are swapped. This is due to a \ac{RISC} feature called} ``branch delay slot''.
\ES{Quizá te preguntes por qué la instrucción de carga (LI) y la instrucción de salto (J o JR) están intercambiadas. Esto se debe a una característica de las arquitecturas \ac{RISC} llamada “branch delay slot”.}
\RU{На самом деле, нам не нужно вникать в эти детали.}
\RU{Нужно просто запомнить: в MIPS инструкция после инструкции перехода исполняется \IT{перед} 
инструкцией перехода.}
\EN{The why this happens is due to quirk in the architecture of some RISC \ac{ISA}s and isn't important for our purposes - we just need to remember that in MIPS, the instruction following a jump or branch instruction
is executed \IT{before} the jump/brunch instruction itself.}
\RU{Таким образом, инструкция перехода всегда поменена местами с той, которая должна быть исполнена перед ней.}
\EN{As a consequence, branch instructions always swap place with the instruction which must be executed beforehand.}
\ES{En realidad, no necesitamos entrar en estos detalles.}
\ES{Solo hay que recordar que en MIPS, la instrucción que sigue a una instrucción de salto se ejecuta \IT{antes} que la propia instrucción de salto/ramificación.}
\ES{Como consecuencia, las instrucciones de salto siempre intercambian su lugar con la instrucción que debe ejecutarse previamente.}
% A footnote/link to http://en.wikipedia.org/wiki/Delay_slot#Branch_delay_slots or
% something similar might be useful for the people more interested in it.

\subsection{\RU{Еще кое-что об именах инструкций и регистров в MIPS}\EN{A note about MIPS instruction/register names}\ES{Una nota sobre los nombres de instrucciones/registros en MIPS}}

\RU{Имена регистров и инструкций в мире MIPS традиционно пишутся в нижнем регистре.}
\EN{Register and instruction names in the world of MIPS are traditionally written in lowercase.}
\RU{Но мы будем использовать верхний регистр, потому что имена инструкций и регистров других \ac{ISA} в этой книге так же в верхнем регистре.}
\EN{However, for the sake of consistency, we'll stick to using uppercase letters, as it is the convention followed by all other \ac{ISA}s featured this book.}
\ES{En el mundo MIPS, los nombres de registros e instrucciones tradicionalmente se escriben en minúsculas.}
\ES{Sin embargo, para mantener la coherencia, usaremos mayúsculas, ya que es la convención seguida por las demás \ac{ISA} presentadas en este libro.}

\fi

\fi

\section{\Conclusion{}}

\RU{Основная разница между кодом x86/ARM и x64/ARM64 в том, что указатель на строку теперь 64-битный.}
\EN{The main difference between x86/ARM and x64/ARM64 code is that pointer to the string is now 64-bits in length.}
\ES{La principal diferencia entre el código x86/ARM y x64/ARM64 es que el puntero a la cadena ahora es de 64 bits.}
\RU{Действительно, ведь для того современные \ac{CPU} и стали 64-битными, потому что подешевела память,
её теперь можно поставить в компьютер намного больше, и чтобы её адресовать, 32-х бит уже
недостаточно.}
\EN{Indeed, modern \ac{CPU}s are now 64-bit due to both the reduced cost of memory and the greater demand for it by modern applications. 
We can add much more memory to our computers than 32-bit pointers are able to address.}
\ES{En efecto, las \ac{CPU} modernas son de 64 bits porque la memoria se abarató y ahora se puede instalar mucha más; para direccionarla, 32 bits ya no bastan.}
\RU{Поэтому все указатели теперь 64-битные.}\EN{As such, all pointers are now 64-bit.}\ES{Por eso todos los punteros son ahora de 64 bits.}

% sections
\ifdefined\IncludeExercises
\input{patterns/01_helloworld/exercises}
\fi
