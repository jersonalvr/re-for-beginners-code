\chapter{\EN{ARM-specific details}\RU{Кое-что специфичное для ARM}\ES{Detalles específicos de ARM}}

\section{\RU{Знак номера}\EN{Number sign} (\#) \RU{перед числом}\EN{before number}\ES{Signo numeral (\#) antes del número}}

\RU{Компилятор Keil, \IDA и objdump предваряет все числа знаком номера (\q{\#}), например:}
\EN{The Keil compiler, \IDA and objdump precede all numbers with the \q{\#} number sign, for example:}
\ES{El compilador Keil, \IDA y objdump anteponen a todos los números el signo \q{\#}, por ejemplo:}
\lstref{Keil_number_sign}.
\RU{Но когда GCC 4.9 выдает результат на языке ассемблера, он так не делает, например:}
\EN{But when GCC 4.9 generates assembly language output, it doesn't, for example: }
\ES{Pero cuando GCC 4.9 genera salida en lenguaje ensamblador, no lo hace, por ejemplo:}
\lstref{GCC_no_number_sign}.

\RU{Так что листинги для ARM в этой книге в каком-то смысле перемешаны.}
\EN{The ARM listings in this book are somewhat mixed.}
\ES{Así que los listados para ARM en este libro están, en cierto sentido, mezclados.}

\RU{Трудно сказать, как правильнее.}\EN{It's hard to say, which method is right.}
\RU{Должно быть, всякий должен придерживаться тех правил, которые приняты в той среде, в которой он работает.}%
\EN{Supposedly, one has to obey the rules accepted in environment he/she works in.}
\ES{Es difícil decir cuál es lo correcto. Probablemente conviene ceñirse a las convenciones aceptadas en el entorno en el que se trabaja.}

% sections
\input{patterns/ARM/post_pre_index.tex}
\section{\RU{Загрузка констант в регистр}\EN{Loading a constant into a register}\ES{Cargar una constante en un registro}}
\label{ARM_big_constants}

\subsection{\RU{32-битный}\EN{32-bit}\ES{32 bits} ARM}
\label{ARM_big_constants_loading}

\RU{Как мы уже знаем, все инструкции имеют длину в 4 байта в режиме ARM и 2 байта в режиме Thumb.}
\EN{Aa we already know, all instructions have a length of 4 bytes in ARM mode and 2 bytes in Thumb mode.}
\ES{Como ya sabemos, todas las instrucciones tienen una longitud de 4 bytes en modo ARM y de 2 bytes en modo Thumb.}
\RU{Как в таком случае записать в регистр 32-битное число, если его невозможно закодировать
внутри одной инструкции?}
\EN{Then how can we load a 32-bit value into a register, if it's not possible to encode it in one instruction?}
\ES{Entonces, ¿cómo podemos cargar un valor de 32 bits en un registro si no es posible codificarlo en una sola instrucción?}

\RU{Попробуем}\EN{Let's try}\ES{Probemos}:

\begin{lstlisting}
unsigned int f()
{
	return 0x12345678;
};
\end{lstlisting}

\begin{lstlisting}[caption=GCC 4.6.3 -O3 \ARMMode]
f:
        ldr     r0, .L2
        bx      lr
.L2:
        .word   305419896 ; 0x12345678
\end{lstlisting}

\RU{Т.е., значение \TT{0x12345678} просто записано в памяти отдельно и загружается, если нужно.}
\EN{So, the \TT{0x12345678} value is just stored aside in memory and loaded if needed.}
\ES{Es decir, el valor \TT{0x12345678} se almacena aparte en memoria y se carga cuando es necesario.}
\RU{Но можно обойтись и без дополнительного обращения к памяти.}
\EN{But it's possible to get rid of the additional memory access.}
\ES{Pero también es posible evitar el acceso adicional a memoria.}

\begin{lstlisting}[caption=GCC 4.6.3 -O3 -march{=}armv7-a (\ARMMode)]
movw    r0, #22136      ; 0x5678
movt    r0, #4660       ; 0x1234
bx      lr
\end{lstlisting}

\RU{Видно, что число загружается в регистр по частям, в начале младшая часть 
(при помощи инструкции MOVW), затем старшая (при помощи MOVT).}
\EN{We see that the value is loaded into the register by parts, the lower part first (using MOVW), 
then the higher (using MOVT).}
\ES{Se ve que el valor se carga en el registro por partes: primero la parte baja (con MOVW) y luego la alta (con MOVT).}

\RU{Следовательно, нужно 2 инструкции в режиме ARM, чтобы записать 32-битное число в регистр.}
\EN{This implies that 2 instructions are necessary in ARM mode for loading a 32-bit value into a register.}
\ES{Por lo tanto, se necesitan 2 instrucciones en modo ARM para cargar un valor de 32 bits en un registro.}
\RU{Это не так уж и страшно, потому что в реальном коде не так уж и много констант (кроме 0 и 1).}
\EN{It's not a real problem, because in fact there are not many constants in real code (except of 0 and 1).}
\ES{No es un gran problema, porque en el código real no hay tantas constantes (excepto 0 y 1).}
\RU{Значит ли это, что это исполняется медленнее чем одна инструкция, как две инструкции?}
\EN{Does it mean that the two-instruction version is slower than one-instruction version?}
\ES{¿Significa eso que la versión de dos instrucciones es más lenta que la de una sola instrucción?}
\RU{Вряд ли, наверняка современные процессоры ARM наверняка умеют распознавать такие 
последовательности и исполнять их быстро.}
\EN{Doubtfully. Most likely, modern ARM processors are able to detect such sequences and execute
them fast.}
\ES{Probablemente no. Lo más seguro es que los procesadores ARM modernos detecten estas secuencias y las ejecuten rápidamente.}

\RU{А \IDA легко распознает подобные паттерны в коде и дизассемблирует эту функцию как:}
\EN{On the other hand, \IDA is able to detect such patterns in the code and disassembles this function as:}
\ES{Por otro lado, \IDA reconoce fácilmente estos patrones en el código y desensambla esta función como:}

\begin{lstlisting}
MOV    R0, 0x12345678
BX     LR
\end{lstlisting}

\subsection{ARM64}

\begin{lstlisting}
uint64_t f()
{
	return 0x12345678ABCDEF01;
};
\end{lstlisting}

\begin{lstlisting}[caption=GCC 4.9.1 -O3]
mov	x0, 61185   ; 0xef01
movk	x0, 0xabcd, lsl 16
movk	x0, 0x5678, lsl 32
movk	x0, 0x1234, lsl 48
ret
\end{lstlisting}

\index{ARM!\Instructions!MOVK}
\TT{MOVK} \RU{означает}\EN{stands for} \q{MOV Keep}, \RU{т.е. она записывает 16-битное значение в регистр, не трогая
при этом остальные биты.}\EN{i.e., it writes a 16-bit value into the register, not touching the rest of the bits.}
\ES{i.e., escribe un valor de 16 bits en el registro sin tocar el resto de los bits.}
\index{ARM!Optional operators!LSL}
\EN{The}\RU{Суффикс} \TT{LSL} \RU{сдвигает значение в каждом случае влево на 16, 32 и 48 бит. Сдвиг происходит
перед загрузкой.}\EN{suffix shifts left the value by 16, 32 and 48 bits at each step. The shifting is done before loading.}
\RU{Таким образом, нужно 4 инструкции, чтобы записать в регистр 64-битное значение.}
\EN{This implies that 4 instructions are necessary to load a 64-bit value into a register.}
\ES{Por lo tanto, se necesitan 4 instrucciones para cargar un valor de 64 bits en un registro.}

\subsubsection{\RU{Записать числа с плавающей точкой в регистр}\EN{Storing floating-point number into register}\ES{Almacenar un número en coma flotante en un registro}}

\RU{Некоторые числа можно записывать в D-регистр при помощи только одной инструкции.}
\EN{It's possible to store a floating-point number into a D-register using only one instruction.}
\ES{Algunos números se pueden escribir en un registro D usando solo una instrucción.}

\RU{Например}\EN{For example}\ES{Por ejemplo}:

\begin{lstlisting}
double a()
{
	return 1.5;
};
\end{lstlisting}

\begin{lstlisting}[caption=GCC 4.9.1 -O3 + objdump]
0000000000000000 <a>:
   0:   1e6f1000        fmov    d0, #1.500000000000000000e+000
   4:   d65f03c0        ret
\end{lstlisting}

\RU{Число $1.5$ действительно было закодировано в 32-битной инструкции.}
\EN{The number $1.5$ was indeed encoded in a 32-bit instruction.}
\ES{El número $1.5$ efectivamente fue codificado en una instrucción de 32 bits.}
\RU{Но как}\EN{But how}?\ES{¿Pero cómo?}
\index{ARM!\Instructions!FMOV}
\RU{В ARM64, инструкцию \TT{FMOV} есть 8 бит для кодирования некоторых чисел с плавающей запятой.}
\EN{In ARM64, there are 8 bits in the \TT{FMOV} instruction for encoding some floating-point numbers.}
\ES{En ARM64, la instrucción \TT{FMOV} dispone de 8 bits para codificar ciertos números en coma flotante.}
\RU{В \cite{ARM64ref} алгоритм называется \TT{VFPExpandImm()}.}
\EN{The algorithm is called \TT{VFPExpandImm()} in \cite{ARM64ref}.}
\ES{El algoritmo se denomina \TT{VFPExpandImm()} en \cite{ARM64ref}.}
\index{minifloat}
\EN{This is also called}\RU{Это также называется}\ES{Esto también se llama} \IT{minifloat}\footnote{\href{http://go.yurichev.com/17139}{wikipedia}}.
\RU{Мы можем попробовать разные: $30.0$ и $31.0$ компилятору удается закодировать, а $32.0$ уже нет, для него
приходится выделять 8 байт в памяти и записать его там в формате IEEE 754:}
\EN{We can try different values: the compiler is able to encode $30.0$ and $31.0$, but it couldn't encode $32.0$,
as 8 bytes have to be allocated for this number in the IEEE 754 format:}
\ES{Podemos probar con distintos valores: el compilador puede codificar $30.0$ y $31.0$, pero no pudo codificar $32.0$, por lo que se deben reservar 8 bytes en memoria y escribirlo allí en formato IEEE 754:}

\begin{lstlisting}
double a()
{
	return 32;
};
\end{lstlisting}

\begin{lstlisting}[caption=GCC 4.9.1 -O3]
a:
	ldr	d0, .LC0
	ret
.LC0:
	.word	0
	.word	1077936128
\end{lstlisting}

\section{\RU{Релоки}\EN{Relocs} \InENRU ARM64\ES{Relocaciones en ARM64}}
\label{ARM64_relocs}

\RU{Как известно, в ARM64 инструкции 4-байтные, так что записать длинное число в регистр одной инструкцией нельзя.}
\EN{As we know, there are 4-byte instructions in ARM64, so it is impossible to write a large number into a register
using a single instruction.}
\ES{Como sabemos, en ARM64 las instrucciones tienen 4 bytes, por lo que es imposible escribir un número grande en un registro con una sola instrucción.}
\RU{Тем не менее, файл может быть загружен по произвольному адресу в памяти, для этого релоки и нужны.}
\EN{Nevertheless, an executable image can be loaded at any random address in memory, so that's why relocs exists.}
\ES{Sin embargo, una imagen ejecutable puede cargarse en cualquier dirección aleatoria de memoria; para eso existen las relocaciones.}
\RU{Больше о них (в связи с Win32 PE)}\EN{Read more about them (in relation to Win32 PE)}\ES{Más sobre ellas (en relación con Win32 PE)}: \myref{subsec:relocs}.

\index{ARM!\Instructions!ADRP/ADD pair}
\RU{В ARM64 принят следующий метод: адрес формируется при помощи пары инструкций: \TT{ADRP} и \ADD.}
\EN{The address is formed using the \TT{ADRP} and \ADD instruction pair in ARM64.}
\ES{En ARM64 se adopta el siguiente método: la dirección se forma mediante el par de instrucciones \TT{ADRP} y \ADD.}
\RU{Первая загружает в регистр адрес 4Kb-страницы, а вторая прибавляет остаток.}
\EN{The first loads a 4Kb-page address and the second one adds the remainder.}
\ES{La primera carga en un registro la dirección de una página de 4 KB y la segunda suma el resto.}
\RU{Скомпилируем пример из}\EN{Let's compile the example from}\ES{Compilaremos el ejemplo de} \q{\HelloWorldSectionName} 
(\lstref{hw_c}) \InENRU GCC (Linaro) 4.9 \RU{под}\EN{under}\ES{en} win32:

\begin{lstlisting}[caption=GCC (Linaro) 4.9 \AndENRU objdump \EN{of object file}\RU{объектного файла}]
...>aarch64-linux-gnu-gcc.exe hw.c -c

...>aarch64-linux-gnu-objdump.exe -d hw.o

...

0000000000000000 <main>:
   0:   a9bf7bfd        stp     x29, x30, [sp,#-16]!
   4:   910003fd        mov     x29, sp
   8:   90000000        adrp    x0, 0 <main>
   c:   91000000        add     x0, x0, #0x0
  10:   94000000        bl      0 <printf>
  14:   52800000        mov     w0, #0x0                        // #0
  18:   a8c17bfd        ldp     x29, x30, [sp],#16
  1c:   d65f03c0        ret

...>aarch64-linux-gnu-objdump.exe -r hw.o

...

RELOCATION RECORDS FOR [.text]:
OFFSET           TYPE              VALUE
0000000000000008 R_AARCH64_ADR_PREL_PG_HI21  .rodata
000000000000000c R_AARCH64_ADD_ABS_LO12_NC  .rodata
0000000000000010 R_AARCH64_CALL26  printf
\end{lstlisting}

\RU{Итак, в этом объектом файле три релока.}
\EN{So there are 3 relocs in this object file.}
\ES{Así que en este archivo objeto hay 3 relocaciones.}

\begin{itemize}
\item 
\RU{Самый первый берет адрес страницы, отсекает младшие 12 бит и записывает оставшиеся старшие 21
в битовые поля инструкции \TT{ADRP}. Это потому что младшие 12 бит кодировать не нужно,
и в ADRP выделено место только для 21 бит.}
\EN{The first one takes the page address, cuts the lowest 12 bits and writes the remaining high 21 bits
to the \TT{ADRP} instruction's bit fields. This is because we don't need to encode the low 12 bits,
and the ADRP instruction has space only for 21 bits.}
\ES{La primera toma la dirección de la página, recorta los 12 bits inferiores y escribe los 21 bits altos restantes en los campos de bits de la instrucción \TT{ADRP}. Esto se debe a que no es necesario codificar los 12 bits bajos y la instrucción ADRP solo tiene espacio para 21 bits.}

\item \RU{Второй ---- 12 бит адреса, относительного от начала страницы, в поля инструкции \ADD.}
\EN{The second one puts the 12 bits of the address relative to the page start into the \ADD instruction's bit fields.}
\ES{La segunda coloca los 12 bits de la dirección relativa al inicio de la página en los campos de bits de la instrucción \ADD.}

\item \RU{Последний, 26-битный, накладывается на инструкцию по адресу \TT{0x10}, где переход на функцию \printf.}
\EN{The last, 26-bit one, is applied to the instruction at address \TT{0x10} where the 
jump to the \printf function is.}
\ES{La última, de 26 bits, se aplica a la instrucción en la dirección \TT{0x10} donde está el salto a la función \printf.}
\RU{Все адреса инструкций в ARM64 (да и в ARM в режиме ARM) имеют нули в двух младших битах
(потому что все инструкции имеют размер в 4 байта),
так что нужно кодировать только старшие 26 бит из 28-битного адресного пространства ($\pm 128$MB).}
\EN{All ARM64 (and in ARM in ARM mode) instruction addresses have zeroes in the two lowest bits
(because all instructions have a size of 4 bytes),
so one need to encode only the highest 26 bits of 28-bit address space ($\pm 128$MB).}
\ES{Todas las direcciones de instrucciones en ARM64 (y en ARM en modo ARM) tienen ceros en los dos bits menos significativos (porque todas las instrucciones tienen un tamaño de 4 bytes), así que solo es necesario codificar los 26 bits superiores del espacio de direcciones de 28 bits ($\pm 128$ MB).}

\end{itemize}

\RU{В слинкованном исполняемом файле релоков в этих местах нет: потому что там уже точно известно, 
где будет находится строка \q{Hello!}, и в какой странице, а также известен адрес функции \puts.}
\EN{There are no such relocs in the executable file: because it's known where the \q{Hello!} string
is located, in which page, and the address of \puts is also known.}
\ES{En el archivo ejecutable enlazado no hay tales relocaciones en estos lugares, porque ya se conoce dónde estará la cadena \q{Hello!}, en qué página, y también se conoce la dirección de la función \puts.}
\RU{И поэтому там, в инструкциях \TT{ADRP}, \ADD и \TT{BL}, уже проставлены нужные значения 
(их проставил линкер во время компоновки):}
\EN{So there are values set already in the \TT{ADRP}, \ADD and \TT{BL} instructions
(the linker has written them while linking):}
\ES{Por lo tanto, los valores ya están establecidos en las instrucciones \TT{ADRP}, \ADD y \TT{BL} (el enlazador los escribió durante el enlace):}

\begin{lstlisting}[caption=objdump \EN{of executable file}\RU{исполняемого файла}]
0000000000400590 <main>:
  400590:       a9bf7bfd        stp     x29, x30, [sp,#-16]!
  400594:       910003fd        mov     x29, sp
  400598:       90000000        adrp    x0, 400000 <_init-0x3b8>
  40059c:       91192000        add     x0, x0, #0x648
  4005a0:       97ffffa0        bl      400420 <puts@plt>
  4005a4:       52800000        mov     w0, #0x0                        // #0
  4005a8:       a8c17bfd        ldp     x29, x30, [sp],#16
  4005ac:       d65f03c0        ret

...

Contents of section .rodata:
 400640 01000200 00000000 48656c6c 6f210000  ........Hello!..
\end{lstlisting}

\index{ARM!\Instructions!BL}
\RU{В качестве примера, попробуем дизассемблировать инструкцию BL вручную.}
\EN{As an example, let's try to disassemble the BL instruction manually.}\ES{Como ejemplo, intentemos desensamblar manualmente la instrucción BL.}\PTBRph{}\PLph{}\\
\TT{0x97ffffa0} \RU{это}\EN{is} $10010111111111111111111110100000b$.
\RU{В соответствии с}\EN{According to}\ES{De acuerdo con} \cite[C5.6.26]{ARM64ref}, 
imm26 \RU{это последние 26 бит}\EN{is the last 26 bits}\ES{son los últimos 26 bits}: 
$imm26 = 11111111111111111110100000$.
\RU{Это}\EN{It is}\ES{Es} \TT{0x3FFFFA0}, \RU{но}\EN{but the}\ES{pero el} \ac{MSB} \RU{это}\EN{is}\ES{es} 1, 
\RU{так что число отрицательное, и в терминах модульной арифметики, по модулю $2^{32}$, это}
\EN{so the number is negative, and in terms of modular arithmetic by modulo $2^{32}$, it is} 
\ES{por lo que el número es negativo y, en términos de aritmética modular módulo $2^{32}$, es} \TT{0xFFFFFFA0}.
\RU{И снова, по модулю}\EN{Again, by modulo}\ES{De nuevo, módulo} $2^{32}$, \TT{0xFFFFFFA0} * 4 = \TT{0xFFFFFE80}, 
\AndENRU \TT{0x4005a0} + \TT{FFFFFE80} = \TT{0x400420} 
(\EN{please note: we consider the address of the BL instruction, 
not the current value of \ac{PC}, which may be different!}
\RU{пожалуйста заметьте: мы берем адрес инструкции BL, 
а не текущее значение \ac{PC}, которое может быть другим!}
\ES{ten en cuenta: consideramos la dirección de la instrucción BL, no el valor actual de \ac{PC}, que puede ser distinto}).
\RU{Так что адрес в итоге}\EN{So the destination address is}\ES{Así que la dirección de destino es} \TT{0x400420}.\\
\\
\RU{Больше о релоках связанных с ARM64}\EN{More about ARM64-related relocs}\ES{Más sobre relocaciones relacionadas con ARM64}: \cite{ARM64_ELF}.

