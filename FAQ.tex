\subsection*{mini-\RU{ЧаВО}\EN{FAQ}\ES{FAQ}}

\newcommand{\HACKINGMdURL}{https://github.com/dennis714/RE-for-beginners/blob/master/HACKING.md}
\newcommand{\FNURLREDDIT}{\footnote{\href{http://go.yurichev.com/17027}{reddit.com/r/ReverseEngineering/}}}

Q: \EN{Why one should learn assembly language these days?}\RU{Зачем в наше время нужно изучать язык ассемблера?}\ES{¿Por qué debería aprenderse el lenguaje ensamblador hoy en día?}\\
A: \EN{Unless you are an \ac{OS} developer, you probably don't need to code in assembly\EMDASH{}modern compilers 
are much better at performing optimizations than humans}
\RU{Если вы не разработчик \ac{OS}, вам наверное не нужно писать на ассемблере:
современные компиляторы оптимизируют код намного лучше человека}%
\ES{A menos que seas desarrollador de un \ac{OS}, probablemente no necesites programar en ensamblador: los compiladores modernos optimizan mucho mejor que los humanos}%
\footnote{\RU{Очень хороший текст на эту тему}\EN{A very good text about this topic}\ES{Un texto muy bueno sobre este tema}: \cite{AgnerFog}}.
\EN{Also, modern \ac{CPU}s are very complex devices and assembly knowledge doesn't really help one to understand their internals.}
\RU{К тому же, современные \ac{CPU} это крайне сложные устройства и знание ассемблера вряд ли
поможет узнать их внутренности.}
\ES{Además, las \ac{CPU} modernas son dispositivos muy complejos y saber ensamblador difícilmente ayuda a comprender su funcionamiento interno.}
\EN{That being said, there are at least two areas where a good understanding of assembly can be helpful: 
First and foremost, security/malware research. It is also a good way to gain a better understanding of your compiled code whilst debugging.}
\RU{Но все-таки остается по крайней мере две области, где знание ассемблера может хорошо
помочь:
1) исследование malware (\IT{зловредов}) с целью анализа; 2) лучшее понимание
вашего скомпилированного кода в процессе отладки.}
\ES{Aun así, hay al menos dos áreas donde comprender bien el ensamblador resulta útil: 1) el análisis de seguridad y malware; 2) entender mejor tu código compilado mientras depuras.}
\EN{This book is therefore intended for those who want to understand assembly language rather 
than to code in it, which is why there are many examples of compiler output contained within.}
\RU{Таким образом, эта книга предназначена для тех, кто хочет скорее понимать ассемблер,
нежели писать на нем, и вот почему здесь масса примеров, связанных с результатами
работы компиляторов.}
\ES{Por ello, este libro está pensado para quienes quieren entender el ensamblador más que programar en él, de ahí la cantidad de ejemplos de salidas de compiladores.}\\
\\
Q: \RU{Я кликнул на ссылку внутри PDF-документа, как теперь вернуться назад?}\EN{I clicked on a hyperlink inside of a PDF-document, how do I get back?}\ES{Hice clic en un hipervínculo dentro de un PDF, ¿cómo vuelvo atrás?}\\
A: \RU{В Adobe Acrobat Reader нажмите сочетание Alt+LeftArrow.}\EN{In Adobe Acrobat Reader click Alt+LeftArrow.}\ES{En Adobe Acrobat Reader pulsa Alt+Flecha Izquierda.}\\
\\
\ifx\LITE\undefined
Q: \RU{Ваша книга слишком большая! Нет ли чего покороче?}\EN{Your book is huge! Is there anything shorter?}\ES{¡Tu libro es enorme! ¿Hay algo más corto?}\\
A: \RU{Есть сокращенная lite-версия}\EN{There is shortened lite version found here}\ES{Hay una versión abreviada (lite) aquí}: \url{http://beginners.re/\#lite}.\\
\\
\fi
Q: \RU{Я не могу понять, стоит ли мне заниматься reverse engineering-ом}\EN{I'm not sure, if I should try to learn reverse engineering or not}\ES{No estoy seguro de si debería aprender ingeniería inversa o no}.\\
A: \RU{Наверное, среднее время для освоения сокращенной LITE-версии\EMDASH{}1-2 месяца.}%
\EN{Perhaps, the average time to become familiar with the contents of the shortened LITE-version is 1-2 month(s).}%
\ES{Quizá el tiempo medio para familiarizarse con la versión abreviada LITE sea de 1 a 2 meses.}\\
\\
Q: \RU{Могу ли я распечатать эту книгу? Использовать её для обучения?}\EN{May I print this book? Use it for teaching?}\ES{¿Puedo imprimir este libro? ¿Usarlo para enseñar?}\\
A: \RU{Конечно, поэтому книга и лицензирована под лицензией Creative Commons.}\EN{Of course! That's why book is licensed under Creative Commons license.}
\EN{One might also want to build one's own version of book\EMDASH{}read \href{\HACKINGMdURL}{here} to find out more.}
\RU{Кто-то может захотеть скомпилировать свою собственную версию книги, читайте \href{\HACKINGMdURL}{здесь} об этом.}
\ES{Por supuesto; por eso el libro está bajo licencia Creative Commons. También puedes compilar tu propia versión del libro; lee \href{\HACKINGMdURL}{aquí} para saber más.}\\
\\
Q: \RU{Я хочу перевести вашу книгу на другой язык}\EN{I want to translate your book to some other language}\ES{Quiero traducir tu libro a otro idioma}.\\
A: \RU{Прочитайте}\EN{Read}\ES{Lee} \href{https://github.com/dennis714/RE-for-beginners/blob/master/Translation.md}{\RU{мою заметку для переводчиков}\EN{my note to translators}\ES{mi nota para traductores}}.\\
\\
Q: \RU{Как можно найти работу reverse engineer-а}\EN{How does one get a job in reverse engineering}? \ES{¿Cómo conseguir trabajo como ingeniero inverso?}\\
A: \RU{На reddit, посвященному RE\FNURLREDDIT, время от времени бывают hiring thread}
\EN{There are hiring threads that appear from time to time on reddit devoted to RE\FNURLREDDIT}
\ES{En reddit dedicado a RE\FNURLREDDIT aparecen hilos de contratación de vez en cuando}
(\href{http://go.yurichev.com/17333}{2013 Q3}, 
\href{http://go.yurichev.com/17334}{2014}).
\RU{Посмотрите там}\EN{Try looking there}.\ES{Échales un ojo.}
\EN{A somewhat related hiring thread can be found in the \q{netsec} subreddit}\RU{В смежном субреддите \q{netsec} имеется похожий тред}\ES{En el subred \q{netsec} hay un hilo parecido}:
\href{http://go.yurichev.com/17335}{2014 Q2}.\\
\\
\RU{Q: Куда пойти учиться в Украине?\\
A: \href{http://go.yurichev.com/17336}{НТУУ \q{КПИ}: \q{Аналіз програмного коду та бінарних вразливостей}};
\href{http://go.yurichev.com/17337}{факультативы}.\\
\\}
\ES{Q: ¿Dónde estudiar en Ucrania?\\
A: \href{http://go.yurichev.com/17336}{NTUU \q{KPI}: \q{Аналіз програмного коду та бінарних вразливостей}};
\href{http://go.yurichev.com/17337}{optativas}.\\
\\}
Q: \EN{I have a question}\RU{У меня есть вопрос}\ES{Tengo una pregunta}...\\
A: \EN{Send it to me by email}\RU{Напишите мне его емейлом}\ES{Envíamela por correo} (\EMAIL).
